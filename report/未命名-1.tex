\documentclass[UTF8]{article}

%--
\usepackage{ctex}
\usepackage{graphicx}
\usepackage[margin=1in]{geometry}

%--
\begin{document}
    
{\flushleft \bf \Large 主题:} Jmtrace实验报告
%--
{\flushleft \bf \Large 姓名:} 黄彬寓

{\flushleft \bf \Large 学号:} MF20330030




    
%=========================================================================
\section{设计总述}
访问内存的指令只有 getstatic/putstatic/getfield/putfield/*aload/*astore,因此我们只要对上述访问内存的指令进行捕获,就能够收集到所有访问共享内存的信息。

本实验使用到的技术主要有ASM和java.lang.Instrument这两个模块,Instrument提供了加载class文件后对文件中的字节码进行修改的功能,而ASM用来具体完成对字节码的重写。

Instrument有两种加载Agent模式,一种是on load加载,也就是在VM启动时加载Agent;第二种是on attach加载,也就是在VM运行时加载Agent,这里我们使用的是启动时加载,所以命令参数为含-javaagent的形式,且需要对premain函数进行重载。

ASM是功能比较齐全的java字节码操作与分析框架,通过ASM框架,我们可以利用访问者设计模式进行遍历,在遍历过程中对字节码进行修改,主要是对visitInsn和visitFieldInsn两个ASM内定义函数进行重写,使得我们希望的内容能够被打印出来。


\section{实现步骤}
1.在Instrument模块,通过premain在启动时加载Agent,JVM在每次在装载class的时候调用transform函数,通过transform我们获得了进行改写的接口。

2.随后到达ClassVisitor模块,我们创建一个它的继承类ASMClassChange,来对它进行重写,通过在其内部重写visitMethod使我们能够进一步地对MethodVisitor模块进行改写。

3.随后进入MethodVisitor模块,我们创建一个它的继承类ASMMethodChange,在这个模块中,我们把需要修改的指令分为两部分,一类是*aload和*astore,即数组类型的访问,在重写的visitInsn中进行打印;另一类是get/put static/field,即对field访问,在重写的visitFieldInsn中进行打印。

\section{实现细节}
1.打印条目:在本实验中,打印函数统一使用printLog(int index,int rw,String name,Object owner,String arrayType),这5个变量分别表示为index数组下标,rw读或写,name对象的类名和变量名,owner为对象标识用于生成哈希值,arrayType为数组类型。
 
2.打印前的保存现场:类似于栈中的函数调用,我们打印前需要往栈顶存入希望的变量,以使打印函数printLog能成功访问到,于是要依次往栈中存放index、rw、name、owner、arrayType这些信息。index和arrayType仅对数组有效,name仅对class field有效。

3.*astore类型提取index:在调用*astore指令,栈信息为..., arrayref, index, value,显然index并不在栈顶,我们通过先执行DUP2使得栈变为..., arrayref, index, value, index, value,再通过一个POP指令,将value pop掉使得栈顶为index。

4.使用类全局变量简化提取信息难度:在*aload和*astore中,我们显然需要arrayType和owner,但这些信息仅通过visitInsn传进来的opcode是无法得到的,但是整个执行过程我们可以看作是串行的,所以我们所需要的信息一定恰好在不久之前被访问过,于是通过类全局变量STATIC\_STATE,current\_owner和arrayType在visitFieldInsn中确定,我们就可以轻松的在任何地方知道当前的这些关键信息。


%--
\end{document}